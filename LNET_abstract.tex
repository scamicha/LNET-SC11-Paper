\documentclass[]{sig-alternate}
\usepackage{graphicx}
\usepackage{epstopdf}
\usepackage{url}
\renewcommand{\topfraction}{0.85}
\renewcommand{\textfraction}{0.1}

\conferenceinfo{TBD}{TBD}
\CopyrightYear{2012}
\crdata{}

\begin{document}

\title{A Study of Lustre Networking Over a 100 Gigabit Wide Area Network with 50 milliseconds of Latency}

\numberofauthors{6}
\author{
\alignauthor Scott Michael\\
	\affaddr{Indiana University}\\
	\affaddr{Bloomington, IN 47408}\\
	\email{scamicha@iu.edu}
\alignauthor Liang Zhen\\
        \affaddr{Whamcloud Inc.}\\
        \affaddr{Danville, CA 94526}\\
        \email{liang@whamcloud.com}
\alignauthor Robert Henschel\\
        \affaddr{Indiana University}\\
        \affaddr{Bloomington, IN 47408}\\
        \email{henschel@iu.edu}
\and
\alignauthor Stephen Simms\\
	\affaddr{Indiana University}\\
	\affaddr{Bloomington, IN 47408}\\
	\email{ssimms@indiana.edu}
\alignauthor Eric Barton\\
        \affaddr{Whamcloud Inc.}\\
        \affaddr{Danville, CA 94526}\\
        \email{eeb@whamcloud.com}
\alignauthor Matthew Link\\
	\affaddr{Indiana University}\\
	\affaddr{Bloomington, IN 47408}\\
	\email{mrlink@indiana.edu}		
}

\maketitle

\begin{abstract}

  As part of the SCinet Research Sandbox at the Supercomputing 2011 conference, Indiana University utilized a
  dedicated 100 Gbps wide area network (WAN) link spanning more than 3500 km (2,175 mi) to demonstrate the
  capabilities of the Lustre high performance parallel file system in a high bandwidth, high latency WAN
  environment. This demonstration functioned as a proof of concept and provided an opportunity to study
  Lustre's performance over a 100 Gbps WAN. To characterize the performance of the network and file system a
  series of benchmarks and tests were undertaken. These included low level iperf network tests, Lustre
  networking (LNET) tests, file system tests with the IOR benchmark, and a suite of real-world applications reading
  and writing to the file system. All of the tests and benchmarks were run over a the WAN link with a latency
  of 50.5 ms. In this article we describe the configuration and constraints of the demonstration and focus in
  on the key findings and discoveries made regarding the Lustre networking layer for this extremely high
  bandwidth and high latency connection. Of particular interest is the relationship between the {\tt peer\_credits}
  and {\tt max\_rpcs\_in\_flight} settings when considering LNET performance.

\end{abstract}

\category{H.3.4}{Information Storage and Retrieval}{Systems and Software}[Distributed systems, Performance evaluation (efficiency and effectiveness)]
\category{C.2.2}{Computer-\linebreak Communication Networks}{Network Protocols}[Protocol architecture (OSI model),
Routing protocols]

\terms{Algorithms, Performance}

\keywords{WAN file systems, Lustre, Data Superconductor}
\vspace{0.25in}

\end{document}
